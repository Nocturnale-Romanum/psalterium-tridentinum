% !TEX TS-program = lualatex
% !TEX encoding = UTF-8

\documentclass[psalterium-tridentinum.tex]{subfiles}

\ifcsname preamble@file\endcsname
  \setcounter{page}{\getpagerefnumber{M-pt30_communia}}
\fi

\begin{document}

\feast{CO}{Commune Sanctorum}
	{Commune Sanctorum}{Commune Sanctorum}{1}{}{}{}{}{}{}
\addcontentsline{toc}{chapter}{Communia}

\feast{APEX}{Commune Apostolorum\\et Evangelistarum\\extra Tempus Paschale}
	{Commune Sanctorum}{Commune Apostolorum extra T. P.}{2}{}
	{}{}{}
	{}
	{}
\addcontentsline{toc}{section}{Commune Apostolorum}

\gscore{APEXIa}{I}{}{Regem Apostolorum\idxnewline(tonus simplex)}
\gscore{APEXH}{H}{}{Aeterna Christi munera\idxnewline{}(Apostolorum)}
\rubric{Feria \Rnum{2} et \Rnum{5}:}
\versiculus{In omnem terram exívit sonus eórum.}{Et in fines orbis terræ verba eórum.}
\rubric{Feria \Rnum{3} et \Rnum{6}:}
\versiculus{Constítues eos príncipes super omnem terram.}{Mémores erunt nóminis tui, Domine.}
\rubric{Feria \Rnum{4} et Sabbato (si non fit de S. Maria):}
\versiculus{Nimis honoráti sunt amíci tui, Dómine.}{Nimis confortátus est principátus eórum.}

\feast{APTP}{Commune Apostolorum\\et Evangelistarum\\Tempore Paschali}
	{Commune Sanctorum}{Commune Apostolorum et Evangelistarum T. P.}{2}{}
	{}{}{}
	{}
	{}
\gscore{APTPI}{I}{}{Regem Apostolorum\idxnewline(Tempore Paschali)}
\gscore{APTPH}{H}{}{Tristes erant Apostoli}
\rubric{Ab Ascensione ad Pentecosten, non dicitur \normaltext{Qui surrexísti a mórtuis} sed \normaltext{Qui scandis super sídera}.}

\rubric{Feria \Rnum{2} et \Rnum{5}:}
\versiculus{Sancti et justi, in Dómino gaudéte, allelúia.}{Vos elégit Deus in hereditátem sibi, allelúia.}
\rubric{Feria \Rnum{3} et \Rnum{6}:}
\versiculus{Lux perpétua lucébit Sanctis tuis, Dómine, allelúia.}{Et ætérnitas témporum, allelúia.}
\rubric{Feria \Rnum{4} et Sabbato (si non fit de S. Maria):}
\versiculus{Lætítia sempitérna super cápita eórum, allelúia.}{Gáudeium et exsultatiónem obtinébunt, allelúia.}


\feast{UMEX}{Commune unius Martyris\\extra Tempus Paschale}
	{Commune Sanctorum}{Commune unius Martyris extra T. P.}{2}{}{}{}{}{}{}
\addcontentsline{toc}{section}{Commune Martyrum}

\gscore{UMEXIa}{I}{}{Regem Martyrum\idxnewline(tonus simplex)}
\gscore{UMEXHa}{H}{}{Deus tuorum militum\idxnewline(tonus prior)}
\rubric{Alter tonus ad libitum:}
\gscore{UMEXHb}{H}{}{Deus tuorum militum\idxnewline(tonus alter)}

\rubric{Feria \Rnum{2} et \Rnum{5}:}
\versiculus{Glória et honóre coronásti eum, Dómine.}{Et constituísti eum super ópera mánuum tuárum.}
\rubric{Feria \Rnum{3} et \Rnum{6}:}
\versiculus{Posuísti, Dómine, super caput ejus.}{Corónam de lápide pretióso.}
\rubric{Feria \Rnum{4} et Sabbato (si non fit de S. Maria):}
\versiculus{Magna est glória ejus in salutári tuo.}{Glóriam et magnum decórem impónes super eum.}

\feast{PMEX}{Commune plurimorum Martyrum\\extra Tempus Paschale}
	{Commune Sanctorum}{Commune plurimorum Martyrum extra T. P.}{2}{}{}{}{}{}{}
	
\gscore{UMEXIa}{I}{}{Regem Martyrum (simplex)}
\gscore{PMEXH}{H}{}{Aeterna Christi munera\idxnewline{}(Martyrum)}
\rubric{Sic semper terminandus est Hymnus prædictus.}

\rubric{Feria \Rnum{2} et \Rnum{5}:}
\versiculus{Lætámini in Dómino et exsultáte, justi.}{Et gloriámini, omnes recti corde.}
\rubric{Feria \Rnum{3} et \Rnum{6}:}
\versiculus{Exsúltent justi in conspéctu Dei.}{Et delecténtur in lætítia.}
\rubric{Feria \Rnum{4} et Sabbato (si non fit de S. Maria):}
\versiculus{Justi autem in perpétuum vivent.}{Et apud Dóminum est merces eórum.}

\feast{MRTP}{Commune unius aut plurimorum Martyrum\\Tempore Paschali}
	{Commune Sanctorum}{Commune unius aut plurimorum Martyrum T. P.}{2}{}{}{}{}{}{}

\gscore{MRTPI}{I}{}{Exsultent in Domino}
\rubric{In Festis unius Martyris Hymnus \normaltext{Deus tuorum militum}, pag.\ \pageref{M-UMEXHa} aut \pageref{UMEXHb}, cum doxologia de tempore; et in Festis plurimorum Martyrum Hymnus \scorename{PMEXH}, pag.\ \pageref{M-PMEXH}, cum sua doxologia.}

\rubric{Versiculi sumuntur ex Communi Apostolorum et Evangelistorum Tempore Paschali.}
	
\feast{CONP}{Commune Confessoris}
	{Commune Sanctorum}{Commune Confessoris}{2}{}{}{}{}{}{}
\thispagestyle{empty}
\addcontentsline{toc}{section}{Commune Confessoris}

\rubric{Extra Tempus Paschale:}
\gscore{COPOIa}{I}{}{Regem Confessorum\idxnewline(tonus simplex)}
\rubric{Tempore Paschali:}
\gscore{COPOId}{I}{}{Regem Confessorum\idxnewline(Tempore Paschali)}

\rubric{Feria \Rnum{2}:}
\gscore{COPOF2H}{H}{}{Iste Confessor (F2)}
\rubric{Feria \Rnum{3}:}
\gscore{COPOF3H}{H}{}{Iste Confessor (F3)}
\rubric{Feria \Rnum{4}:}
\gscore{COPOF4H}{H}{}{Iste Confessor (F4)}
\rubric{Feria \Rnum{5}:}
\gscore{COPOF5H}{H}{}{Iste Confessor (F5)}
\rubric{Feria \Rnum{6}:}
\gscore{COPOF6H}{H}{}{Iste Confessor (F6)}
\rubric{Sabbato:}
\gscore{COPOF7H}{H}{}{Iste Confessor (Sabb)}

\rubric{Feria \Rnum{2} et \Rnum{5}:}
\versiculustpall{Amávit eum Dóminus, et ornávit eum.}{Stolam glóriæ índuit eum.}
\rubric{Feria \Rnum{3} et \Rnum{6}, pro Confessore Pontifice:}
\versiculustpall{Elégit eum Dóminus sacerdótem sibi.}{Ad sacrificándum ei hóstiam laudis.}
\rubric{Feria \Rnum{3} et \Rnum{6}, pro Confessore non Pontifice:}
\versiculustpall{Os justi meditábitur sapiéntiam.}{Et lingua ejus loquétur justítiam.}
\rubric{Feria \Rnum{4} et Sabbato (si non fit de S. Maria), pro Confessore Pontifice:}
\versiculustpall{Tu es sacérdos in ætérnum.}{Secúndum órdinem Melchísedech.}
\rubric{Feria \Rnum{4} et Sabbato, pro Confessore non Pontifice:}
\versiculustpall{Lex Dei ejus in corde ipsíus.}{Et non supplantabúntur gressus ejus.}



\feast{MU}{Commune Virginum aut non Virginum}
	{Commune Sanctorum}{Commune Virginum aut non Virginum}{2}{}{}{}{}{}{}
\addcontentsline{toc}{section}{Commune Virginum aut non Virginum}

\rubric{Pro Virgine, extra Tempus Paschale:}
\gscore{MUVXIa}{I}{}{Regem Virginum\idxnewline(tonus simplex)}
\rubric{Pro Virgine, Tempore Paschali:}
\gscore{MUVXId}{I}{}{Regem Virginum\idxnewline(Tempore Paschali)}
\rubric{Pro una non Virgine:}
\gscore{MUNXIa}{I}{}{Laudemus\idxnewline(unius non Virginis)}
\rubric{Pro plurimis non Virginis:}
\gscore{MUNXIb}{I}{}{Laudemus (plurimarum\idxnewline{}non Virginum)}
\rubric{Pro Virgine non Martyre, strophæ 1, 4 et 5 tantum dicuntur.\\Pro non Virgine, strophæ 4 et 5 tantum dicuntur.}
\gscore{MUVMHa}{H}{}{Virginis proles\idxnewline{}(tonus prior)}
\rubric{Alter tonus ad libitum:}
\gscore{MUVMHb}{H}{}{Virginis proles\idxnewline{}(tonus alter)}

\rubric{Feria \Rnum{2} et \Rnum{5}:}
\versiculustpall{Spécie tua et pulchritúdine tua.}{Inténde, próspere procéde, et regna.}
\rubric{Feria \Rnum{3} et \Rnum{6}:}
\versiculustpall{Adjuvábit eam Deus vultu suo.}{Deus in médio ejus, non commovébitur.}
\rubric{Feria \Rnum{4} et Sabbato (si non fit de S. Maria):}
\versiculustpall{Elégit eam Deus, et præelégit eam.}{In tabernáculo suo habitáre facit eam.}

%%%% UNCOMMENT THIS IF WE CONSIDER THAT DAYS WITHIN OCTAVES CAN BE SIMPLEX

% \feast{CDED}{Commune Dedicationis Ecclesiæ}
	% {Commune Sanctorum}{Commune Dedicationis Ecclesiæ}{2}{}{}{}{}{}{}
% \addcontentsline{toc}{section}{Commune Dedicationis}

% \gscore{CDEDI}{I}{}{Domum Dei decet}
% \gscore{CDEDH}{H}{}{Coelestis urbs Jerusalem}

% \rubric{Feria \Rnum{2} et \Rnum{5}:}
% \versiculustpall{Dómum tuam, Dómine, decet sanctitúdo.}{In longitúdinem diérum.}
% \rubric{Feria \Rnum{3} et \Rnum{6}:}
% \versiculustpall{Dómus mea.}{Domus oratiónis vocábitur.}
% \rubric{Feria \Rnum{4} et Sabbato (si non fit de S. Maria):}
% \versiculustpall{Hic est domus Dómini fírmiter ædificáta.}{Bene fundáta est supra firmam petram.}


\feast{CBMV}{Commune Beatæ Mariæ Virginis}
	{Commune Sanctorum}{Commune Beatæ Mariæ Virginis}{2}{}{}{}{}{}{}
\addcontentsline{toc}{section}{Commune Beatæ Mariæ Virginis}

\gscore{CBMVI}{I}{}{Sancta Maria}
\gscore{CBMVH}{H}{}{Quem terra pontus aethera}

\rubric{Feria \Rnum{2} et \Rnum{5}:}
\versiculustpall{Spécie tua et pulchritúdine tua.}{Inténde, próspere procéde, et regna.}
\rubric{Feria \Rnum{3} et \Rnum{6}:}
\versiculustpall{Adjuvábit eam Deus vultu suo.}{Deus in médio ejus, non commovébitur.}
\rubric{Feria \Rnum{4} et Sabbato (si non fit Officium de S. Maria in Sabbato):}
\versiculustpall{Elégit eam Deus, et præelégit eam.}{In tabernáculo suo habitáre facit eam.}

\feast{CSMS}{De Sancta Maria in Sabbato}
	{Commune Sanctorum}{De Sancta Maria in Sabbato}{2}{}
	{}{}{Sabbato Sanctae Mariae}
	{}{}
\addcontentsline{toc}{section}{De Sancta Maria in Sabbato}

\gscore{CSMSI}{I}{}{Ave Maria}
\rubric{Hymnus \scorename{CBMVH}, ut in Communi B.M.V. pag.\ \pageref{M-CBMVH}.}
\versiculustpall{Diffúsa est grátia in lábiis tuis.}{Proptérea benedíxit te Deus in ætérnum.}

\end{document}