% !TEX TS-program = lualatex
% !TEX encoding = UTF-8

\documentclass[psalterium-tridentinum.tex]{subfiles}

\ifcsname preamble@file\endcsname
  \setcounter{page}{\getpagerefnumber{M-pt20_psalterium_ordinarium}}
\fi

\begin{document} %% THIS SUBFILE BEGINS ON A LEFT, EVEN PAGE

\null\thispagestyle{empty}\newpage %% REMOVE THIS IF THIS SUBFILE BEGINS ON A RIGHT, ODD PAGE

\feast{OR}{Ordinarium Divini Offici\\ ad Matutinum}
	{Ordinarium}{Ordinarium}{1}{}{}{}{}{}{}
\addcontentsline{toc}{chapter}{Ordinarium Divini Officii ad Matutinum}

\intermediatetitle{Orationes ante Divinum Officium}
\thispagestyle{empty}

\begin{multicols}{2}
\lettrine{A}{peri}, Dómine, os meum ad benedicéndum nomen sanctum tuum:
munda quoque cor meum ab ómnibus vanis, pervérsis et aliénis cogitatiónibus; 
intelléctum illúmina, afféctum inflámma, ut digne, atténte ac devóte
hoc officium recitáre váleam, et exaudíri mérear
ante conspéctum divinæ Majestátis tuæ. Per Christum Dóminum nostrum. Amen.

~

\lettrine{D}{ómine}, in unióne illíus divínæ intentiónis,
qua ipse in terris laudes Deo persolvísti,
has tibi horas \rubric{(vel} hanc tibi horam\rubric{)} persólvo.
\end{multicols}

\sep

\begin{multicols}{2}
\lettrine{P}{ater noster}, qui es in cælis, sanctificétur nomen tuum.
Advéniat regnum tuum. Fiat volúntas tua, sicut in cælo et in terra.
Panem nostrum quotidiánum da nobis hódie.
Et dimítte nobis débita nostra, sicut et nos dimíttimus debitóribus nostris.
Et ne nos indúcas in tentatiónem: sed líbera nos a malo. Amen.

\vspace{\baselineskip}
\lettrine{A}{ve María}, grátia plena, Dóminus tecum:
benedícta tu in muliéribus, et benedíctus fructus ventris tui Jesus.
Sancta María, Mater Dei, ora pro nobis peccatóribus,
nunc et in hora mortis nostræ. Amen.

\vspace{\baselineskip}
\lettrine{C}{redo in Deum}, Patrem omnipoténtem, Creatórem cæli et terræ.
Et in Jesum Christum, Fílium ejus únicum, Dóminum nostrum,
qui concéptus est de spíritu Sancto, natus ex María Virgine,
passus sub Póntio Piláto, crucifixus, mórtuus et sepúltus:
descéndit ad ínferos: tértia die resurréxit a mórtuis;
ascéndit ad cælos, sedet ad déxteram Patris omnipoténtis:
inde ventúrus est judicáre vivos et mórtuos.
Credo in Spíritum sanctum, sanctam Ecclésiam cathólicam,
Sanctórum communiónem, remissiónem peccatórum,
carnis resurrectiónem, vitam ætérnam. Amen.
\end{multicols}

\pagebreak

\intermediatetitle{Incipit}

\rubric{In Dominicis:}
\vspace{\baselineskip}
\gscore{ORIa}{T}{}{Domine labia mea!Tonus festivus}
\newpage

\rubric{In diebus trium lectionum:}
\vspace{\baselineskip}
\gscore{ORIc}{T}{}{Domine labia mea!Tonus ferialis}
\newpage

\intermediatetitle{Invitatorium}

\rubric{Postea dicitur conveniens Invitatorium, quod ante Psalmum bis cantatur,
et ad singulos ejusdem Psalmi versus
vel integrum vel dimidiatum ab asterisco \normaltext{\GreSpecial{*}} repetitur.}

\label{ORIP2d}
\label{ORIP3c}
\label{ORIP3e}
\label{ORIP4e}
\label{ORIP4g}
\label{ORIP4d}
\label{ORIP5g}
\label{ORIP6a}
\label{ODEFIP}
\label{ORIP6f}
\label{ORIP7a}
\label{ORIP7g}

\psalmus{94}{VLrepet}

\vspace{0.5\baselineskip}

\intermediatetitle{Hymnus}

\vspace{0.6\baselineskip}

\begin{multicols}{2}

\rubric{A Nativitate Domini usque ad Epiphaniam, per Octavam Corporis Christi, 
in Festis Beatæ Mariæ Virginis et per Octavas eorum, etiam tempore Paschali,
in fine omnium Hymnorum ejusdem metri, nec habeant ultimum Versum proprium, dicitur:}

Glória tibi Dómine,\\
Qui natus es de Vírgine,\\
Cum Patr\textit{e} et Sancto Spíritu,\\
In sempitérna sǽcula.

\rubric{Per Octavam Epiphaniæ, et per Octavam Transfigurationis Domini, si de ea fit Officio, 
in fine omnium Hymnorum ejusdem metri, nec habeant ultimum Versum proprium, dicitur:}

Glória tibi Dómine,\\
Qu\textit{i} apparuísti hódie,\\
Cum Patr\textit{e} et Sancto Spíritu,\\
In sempitérna sǽcula.

\rubric{A Dominica in Albis usque ad Ascensionem, 
in fine omnium Hymnorum ejusdem metri, nec habeant ultimum Versum proprium, dicitur:}

Glória tibi Dómine,\\
Qui surrexíst\textit{i} a mórtuis,\\
Cum Patr\textit{e} et Sancto Spíritu,\\ 
In sempitérna sǽcula.

\rubric{Ab Ascensione usque ad Pentecostem, 
in fine omnium Hymnorum ejusdem metri, nec habeant ultimum Versum proprium, dicitur:}

Glória tibi Dómine,\\
Qui scandis super sídera,\\
Cum Patr\textit{e} et Sancto Spíritu,\\
In sempitérna sǽcula.

\end{multicols}

\intermediatetitle{Antiphonæ et Psalmi}

\rubric{Antiphonæ et Psalmi ut in Psalterio. Versus in Officio de Tempore ut in Psalterio, in Festis Sanctorum ut in Proprio vel Communi. Versus cantatur hoc modo, et item respondetur:}

\gscore[n]{ORW}{T}{}{Versus}

\rubric{Post Versum cujuslibet Nocturni dicitur:}
\gscore[n]{ORPN}{T}{}{Pater Noster}

\intermediatetitle{Absolutiones}

\rubric{In \Rnum{1} Nocturno Dominicæ, et Feriæ \Rnum{2} et \Rnum{5}, etiam si Festum occurat:}

\rubric{\emph{Absolutio 1.}}
Exáudi, Dómine Jesu Christe, preces servórum tuórum,~\GreSpecial{+}
et mise\textit{rére} \textbf{no}bis:~\GreSpecial{*}
Qui cum Patre et Spíritu Sancto vivis et regnas in sǽcula sæculórum.
\hspace{\specialcharhsep}\rr Amen.

\rubric{In \Rnum{2} Nocturno Dominicæ, et Feriæ \Rnum{3} et \Rnum{6}, etiam si Festum occurat:}

\rubric{\emph{Absolutio 2.}}
Ipsíus píetas et misericódi\textit{a nos} \textbf{ád}juvet,~\GreSpecial{*}
qui cum Patre et Spíritu Sancto vivit et regnat in sǽcula sæculórum.
\hspace{\specialcharhsep}\rr Amen.

\rubric{In \Rnum{3} Nocturno Dominicæ, et Feria \Rnum{4} et Sabbato, etiam si Festum occurat:}

\rubric{\emph{Absolutio 3.}}
A vínculis peccató\textit{rum nos}\textbf{tró}rum~\GreSpecial{*}
absólvat nos omnípotens et miséricors Dóminus.
\hspace{\specialcharhsep}\rr Amen.

\rubric{In Officio B.M.V. in Sabbato:}

\rubric{\emph{Absolutio.}}
Précibus et méritis beátæ Maríæ semper Virginis
et ómni\textit{um San}\textbf{ctó}rum,~\GreSpecial{*}
perdúcat nos Dóminus ad regna cælórum.
\hspace{\specialcharhsep}\rr Amen.

\intermediatetitle{Benedictiones}

\rubric{In \Rnum{1} Nocturno Dominicæ, et Feriæ \Rnum{2} et \Rnum{5}:}

\rubric{\emph{Benedictio 1.}} Benedictió\textit{ne} \textit{per}\textbf{pé}tua~\GreSpecial{*}
benedícat nos Pater ætérnus.
\hspace{\specialcharhsep}\rr Amen.

\rubric{\emph{Benedictio 2.}} Unigénitus \textit{Dei} \textbf{Fí}lius~\GreSpecial{*}
nos benedícere et adjuváre dignétur.
\hspace{\specialcharhsep}\rr Amen.

\rubric{\emph{Benedictio 3.}} Spíritus \textit{Sancti} \textbf{grá}tia~\GreSpecial{*}
illúminet sensus et corda nostra.
\hspace{\specialcharhsep}\rr Amen.

\rubric{In \Rnum{2} Nocturno Dominicæ, et Feriæ \Rnum{3} et \Rnum{6}:}

\rubric{\emph{Benedictio 1.}} Deus Pa\textit{ter om}\textbf{ní}potens~\GreSpecial{*}
sit nobis propítius et clemens.
\hspace{\specialcharhsep}\rr Amen.

\rubric{\emph{Benedictio 2.}} Chris\textit{tus per}\textbf{pé}tuæ~\GreSpecial{*}
det nobis gaúdia vitæ.
\hspace{\specialcharhsep}\rr Amen.

\rubric{\emph{Benedictio 3.}} Ignem su\textit{i a}\textbf{mó}ris~\GreSpecial{*}
accéndat Deus in córdibus nostris.
\hspace{\specialcharhsep}\rr Amen.

\rubric{In \Rnum{3} Nocturno Dominicæ, et in Feriis, quando legitur Homilia cum Evangelio:}

\rubric{\emph{Benedictio 1.}}
Evangé\textit{lica} \textbf{léc}tio~\GreSpecial{*}
sit nobis salus et protéctio.
\hspace{\specialcharhsep}\rr Amen.

\rubric{\emph{Benedictio 2.}}
Diví\textit{num au}\textbf{xí}lium~\GreSpecial{*}
máneat semper nobíscum.
\hspace{\specialcharhsep}\rr Amen.

\rubric{\emph{Benedictio 3.}}
Ad societátem cívium \textit{super}\textbf{nó}rum~\GreSpecial{*}
perdúcat nos Rex Angelórum.
\hspace{\specialcharhsep}\rr Amen.

\rubric{Feria \Rnum{4} et Sabbato:}

\rubric{\emph{Benedictio 1.}}
Ille nos \textit{bene}\textbf{dí}cat,~\GreSpecial{*}
qui sine fine vivit et regnat.
\hspace{\specialcharhsep}\rr Amen.

\rubric{\emph{Benedictio 2.}}
Diví\textit{num au}\textbf{xí}lium~\GreSpecial{*}
máneat semper nobíscum.
\hspace{\specialcharhsep}\rr Amen.

\rubric{\emph{Benedictio 3.}}
Ad societátem cívium \textit{super}\textbf{nó}rum~\GreSpecial{*}
perdúcat nos Rex Angelórum.
\hspace{\specialcharhsep}\rr Amen.

\rubric{In Festis Sanctorum:}

\rubric{\emph{Benedictio 1.}}
Ille nos \textit{bene}\textbf{dí}cat,~\GreSpecial{*}
qui sine fine vivit et regnat.
\hspace{\specialcharhsep}\rr Amen.

\rubric{\emph{Benedictio 2.}}
Cujus \rubric{(vel} Quarum\rubric{)} \textit{festum} \textbf{có}limus,~\GreSpecial{*}
ipse \rubric{(vel} ipsa \rubric{aut} ipsæ\rubric{)}
intercédat \rubric{(vel} intercédant\rubric{)} pro nobis ad Dóminum.
\hspace{\specialcharhsep}\rr Amen.\\
\rubric{sed si Festum fit de B.M.V.:}\\
\rubric{\emph{Benedictio 2.}}
Cujus \textit{festum} \textbf{có}limus,~\GreSpecial{*}
ipsa Virgo vírginum intercédat pro nobis ad Dóminum.
\hspace{\specialcharhsep}\rr Amen.

\rubric{\emph{Benedictio 3.}}
Ad societátem cívium \textit{super}\textbf{nó}rum~\GreSpecial{*}
perdúcat nos Rex Angelórum.
\hspace{\specialcharhsep}\rr Amen.

\rubric{In Officio B.M.V. in Sabbato:}

\rubric{\emph{Benedictio 1.}}
Nos cum \textit{prole} \textbf{pi}a~\GreSpecial{*}
benedícat Virgo María.
\hspace{\specialcharhsep}\rr Amen.

\rubric{\emph{Benedictio 2.}}
Ipsa \textit{Virgo} \textbf{vír}ginum~\GreSpecial{*}
intercédat pro nobis ad Dóminum.
\hspace{\specialcharhsep}\rr Amen.

\rubric{\emph{Benedictio 3.}}
Per Vír\textit{ginem} \textbf{ma}trem~\GreSpecial{*}
concédat nobis Dóminus salútem et pacem.
\hspace{\specialcharhsep}\rr Amen.

\intermediatetitle{Te Deum}

\rubric{Post ultimam Lectionem, in Dominicis, in Festis et per Octavas, dicitur Hymnus Ambrosianus, pag.\ \pageref{M-ORTDb}. In Adventu autem, et a Dominica Septuagesimæ usque ad Sabbatum sanctum inclusive, non dicitur nisi in Festis.}

\intermediatetitle{Conclusio}

\rubric{Dicto \normaltext{Te Deum}, aut ultimo Responsorio,
statim incipiuntur Laudes a Versu \normaltext{Deus, in adjutórium}.
Si non sequent Laudes, post Hymnum \normaltext{Te Deum},
vel post ultimum Responsorium, dicitur:}

{\grechangedim{spaceabovelines}{-1.5mm}{scalable}\grechangedim{spacebeneathtext}{1.5mm}{scalable}
\rubric{In Dominicis:}
\gscore[n]{ORDV}{T}{}{Dominus vobiscum}
\rubric{In diebus trium lectionum:}
\gscore[n]{ORDVb}{T}{}{Dominus vobiscum}

\rubric{Hic versus non dicitur ab eo, qui non est saltem in ordine diaconatus;
sed ejus loco substituitur:}

\rubric{In Dominicis:}
\gscore[n]{ORDE}{T}{}{Domine exaudi}
\pagebreak
\rubric{In diebus trium lectionum:}
\gscore[n]{ORDEb}{T}{}{Domine exaudi}

}% end grechangedim spaceabovelines/spacebeneathtext

\rubric{Deinde cantatur oratio et respondetur \normaltext{Amen.}}

\versiculus{Dóminus vobíscum.}{Et cum spíritu tuo.}
\rubric{vel \normaltext{Dómine, exáudi}, etc.}

\rubric{\normaltext{Benedicámus Dómino} secundum diem, pag.\ \pageref{M-TCBD}.}
\versiculus{Fidélium ánimæ per misericórdiam Dei requiéscant in pace.}{Amen.}

\rubric{Deinde dicitur \normaltext{Pater noster} totum secreto.}

\vfill
\sep
\vfill

\intermediatetitle{Post Divinum Officium}

\begin{multicols}{2}
\lettrine{S}{acrosánctæ} et indivíduæ Trinitáti, crucifíxi Dómini nostri Jesu Christi humanitáti, beatíssimæ et gloriosíssimæ sempérque Vírginis Maríæ fœcúndæ integritáti, et ómnium Sanctórum universitáti sit sempitérna laus, honor, virtus et glória ab omni creatúra, nobísque remíssio ómnium peccatórum, per infiníta sǽcula sæculórum. Amen.
\end{multicols}

\pagebreak

\feast{TC}{Toni Communes}
	{Toni Communes}{Toni Communes}{1}{}{}{}{}{}{}
\addcontentsline{toc}{chapter}{Toni communes}

\feast{TCDS}{Ultimum Responsorium\\in Dominicis per Annum}
	{Duo Seraphim}{Duo Seraphim}{2}{}{}{}{}{}{}

\gscore{E2N3R2}{R}{8}{Duo Seraphim}

\gresetnabc{1}{invisible}
\feast{TCTD}{Hymnus Ambrosianus}
	{Toni Communes}{Hymnus Ambrosianus}{2}{}{}{}{}{}{}

\gscore{ORTDb}{H}{}{Te Deum laudamus\idxnewline(tonus simplex)}

\gresetnabc{1}{visible}

\vfill
\sep
\vfill

\feast{TCBD}{Toni Communes ad «Benedicamus Domino»}
	{Toni Communes}{ad «Benedicamus Domino»}{2}{}{}{}{}{}{}
\thispagestyle{empty}

{\grechangedim{baselineskip}{55pt plus 1pt minus 5pt}{scalable}
	\rubric{In Dominicis Adventus et Quadragesimæ}
	\gscore{ORBDm}{T}{}{}
\pagebreak
	\rubric{In Dominicis per Annum}
	\gscore{ORBDe}{T}{}{}
	\rubric{Tempore Paschali}
	\gscore{ORBDk}{T}{}{}
	\rubric{In Festis Simplicibus}
	\gscore{ORBDf}{T}{}{}
	\rubric{In Officio Beatæ Mariæ Virginis in Sabbato}
	\gscore{ORBDg}{T}{}{}
	\rubric{In Feriis}
	\gscore{ORBDi}{T}{}{}
}

\end{document}