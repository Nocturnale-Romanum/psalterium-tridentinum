% !TEX TS-program = lualatex
% !TEX encoding = UTF-8

\documentclass[psalterium-tridentinum.tex]{subfiles}

\ifcsname preamble@file\endcsname
  \setcounter{page}{\getpagerefnumber{M-pt10_temporale}}
\fi

\begin{document}

\feast{PT}{Proprium de Tempore}{Proprium de Tempore}{Proprium de Tempore}{1}{}{}{}{}{}{}
\addcontentsline{toc}{chapter}{Proprium de Tempore}

\feast{A1F1}{Tempus Adventus}
	{Proprium de Tempore}{Tempus Adventus}{2}{}
	{}{}{}
	{}
	{}
\rubric{Dominicæ \Rnum{1} et \Rnum{2}}
\gscore{A1F1I}{I}{}{Regem venturum Dominum!Dominicis}
\rubric{In Feriis Hebdomadæ \Rnum{1} et \Rnum{2}}
\gscore{A1F2I}{I}{}{Regem venturum Dominum!Feriis}
\rubric{Dominicæ \Rnum{3} et \Rnum{4}}
\gscore{A3F1I}{I}{}{Prope est jam Dominus!Dominicis}
\rubric{In Feriis Hebdomadæ \Rnum{3} et \Rnum{4}}
\gscore{A3F1I}{I}{}{Prope est jam Dominus!Feriis}
\gscore{A1H}{H}{}{Verbum supernum prodiens}

\feast{1224}{In Vigilia Nativitatis Domini}
	{Proprium de Tempore}{In Vigilia Nativitatis Domini}{3}{24 Decembris}
	{}{}{Jesu Christi, Domini nostri!Nativitas, vigilia}
	{}
	{}
\gscore{1224I}{I}{}{Hodie scietis}

\rubric{Si Vigilia venerit in Dominica, in \Rnum{3} Nocturno versus ut infra, alii de Dominica. Si venerit in Feria, versus ut infra.}

\versiculus{Hódie sciétis quia véniet Dóminus.}{Et mane vidébitis glóriam ejus.}

%% TODO : doxologie propre nativité 
Jesu tibi sit glo'ria,
Qui natus es de Vi'rgine,
Cum Patre et almo Spi'ritu.
In sempiterna sǽcula.

"Sic terminantur omnes Hymni ejusdem metri usque ad Epiphaniam."

%% TODO : doxo propre épiph
Jesu, tibi sit gloria. Qui apparuisti Gentibus Cum Patre, et almo Spi¬ ritu,
In sempiterna saecula Arnen.

"Sic terminantur Hymni per totam Octavam"

Feria quarta cinerum : "In hac et sequentibus Feriis omnia dicuntur ut in Psalterio per annum."



%%% TODO : F2 in TP Invit. Alleluia x3 ?

\end{document}