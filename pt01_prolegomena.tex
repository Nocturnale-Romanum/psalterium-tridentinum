% !TEX TS-program = lualatex
% !TEX encoding = UTF-8

\documentclass[psalterium-tridentinum.tex]{subfiles}

\ifcsname preamble@file\endcsname
  \setcounter{page}{\getpagerefnumber{M-pt01_prolegomena}}
\fi

\begin{document}

\begin{titlepage}
\begin{center}
\null\vspace{5mm}
{\Large\sc{}Nocturnale Romanum}

\vspace{5mm}

{\large\sc{}Tomus IIt}

\vspace{3.5cm}

{\Huge{}PSALTERIUM}

\vspace{1cm}

{\Large\sc{}pro nocturnis horis\\in dominicis, feriis et festis simplicibus}

\vspace{5mm}

{\large\sc{}secundum ordinem Divini Officii\\a Pio pp V restituti}

\vspace{5mm}

{\large\sc{}cum Antiphonis, Invitatoriis et Hymnis\\in cantu gregoriano}

\vspace{3.5cm}

{\large\sc{}instrumentum laboris}

\vfill

MMXXV

\end{center}
\end{titlepage}

\feast{OR}{Proœmium}
	{Proœmium}{Proœmium}{2}{}{}{}{}{}{}
\thispagestyle{empty}
\addcontentsline{toc}{section}{Proœmium}

\begin{paracol}[1]*{2}

\begin{english}
	
\intermediatetitle{On the preservation of fire}

{\setlength{\parindent}{5mm}\small

TODO

}

\end{english}

\switchcolumn

\intermediatetitle{Garder la flamme}

{\setlength{\parindent}{5mm}\small

TODO

}

\switchcolumn*

\begin{english}
	
\intermediatetitle{On the \emph{Ordo} being proper}

{\setlength{\parindent}{5mm}\small

TODO

}

\end{english}

\switchcolumn

\intermediatetitle{Pourquoi un \emph{Ordo} propre?}

{\setlength{\parindent}{5mm}\small

TODO

}

\switchcolumn*

\begin{english}

\intermediatetitle{On the rhythm of Gregorian Chant\\as considered in this edition}

{\setlength{\parindent}{5mm}\small

This Psalter is intended for all those who wish to sing the night hour of Divine Office, 
according to the ancient customs of the Roman ritual family, whatever their approach to rhythm.

To this end, the editors wish to briefly expose the principles that have informed 
the correspondence between neumatic notation and horizontal \emph{episemata} in this book.

In Gregorian Chant, notes have a base length, 
henceforth the \emph{syllabic value}. 
It is the value of a syllable sung on a single note, when cantillation has the text sung on one note for each syllable.

When the neume for a syllable sung on a single note receives an \emph{episema}, or a letter T (\emph{tenete}), 
this note takes on the \emph{long value}, that is longer than the syllabic value. 
In this case, the square note receives an episema as well in this edition.

Notes within a melisma (that is, a syllable sung on more than one note) receive by default the \emph{short value}, 
a value that is shorter than the syllabic value. 
Frequently, their neumes receive an \emph{episema}, or an angled shape distinct from the usual one, or a neumatic break where the complex neume disintegrates into several simpler shapes, 
in which case the notes receive the syllabic value.
In this case, the square notes receive an episema as well, except notes before a \emph{quilisma}: 
it is common knowledge that those are to be somewhat lengthened even if they are not marked with an \emph{episema}.

{\gresetnabc{1}{visible}
\gresetclef{invisible}
\gresetinitiallines{0}
\begin{center}
\begin{minipage}[c]{0.7\textwidth}
\gregorioscore{\subfix{nocturnale-romanum/gabc/rhythmica_en}}
\end{minipage}
\end{center}
}

~

The proportion between the \emph{short value} and the \emph{syllabic value}, 
and between the \emph{syllabic value} and the \emph{long value}, should be consistent, but is at the cantor's discretion. 

It should also take into account the nature of the note: a \emph{stropha} marked with an \emph{episema} receives a 
syllabic value somewhat shorter than that of a \emph{virga} also marked with an \emph{episema}, 
the \emph{stropha} itself being slightly shorter than the \emph{virga}.

Finally, the end of a musical or textual phrase naturally brings about a certain lengthening of the notes all while the sound diminishes. 
This is customarily indicated by the \emph{punctum mora} or \emph{mora} dot.

}

\end{english}

\switchcolumn

\intermediatetitle{Le rythme du chant grégorien\\tel qu'il a servi à préparer cette édition}

{\setlength{\parindent}{5mm}\small

Ce psautier s'adresse à toutes les personnes qui désirent chanter l'heure nocturne de l'Office divin 
selon l'antique coutume de la famille rituelle romaine --- quelle que soit leur approche du rythme.

Pour cela, les éditeurs souhaitent exposer brièvement les principes qui, dans cet ouvrage, 
ont informé la correspondance entre la notation neumatique antique et l'usage désormais bien connu des épisèmes horizontaux.   

Dans le chant grégorien, les notes ont une longueur de base, qu’on peut appeler \emph{valeur syllabique}. 
Celle-ci est la valeur d’une syllabe chantée sur une seule note --- lorsque le texte de la mélodie est cantillé, une note sur chaque syllabe. 

Quand le signe neumatique associé à une syllabe chantée sur une seule note, a reçu dans les manuscrits un épisème, ou la lettre T (\emph{tenete}), 
cette note a une valeur longue, plus longue que la valeur syllabique. 
Dans notre publication, un épisème horizontal est ajouté à la note carrée.

Au sein d’un mélisme, c’est-à-dire au sein d’un enchaînement de plusieurs notes sur la même syllabe, 
les notes ont une valeur plus courte que la valeur syllabique. 
Souvent, les signes neumatiques qui les transcrivent, ont reçu dans les manuscrits un épisème, 
ou bien leur forme a été modifiée par rapport à l’usage habituel, ou encore on constate une «coupure» neumatique; 
dans ces cas, les notes concernées ont une valeur syllabique, 
et notre publication ajoute des épisèmes horizontaux aux notes carrées concernées, 
sauf pour les notes qui précèdent un quilisma, dont on sait qu’elles sont toujours légèrement allongées.

{\gresetnabc{1}{visible}
\gresetclef{invisible}
\gresetinitiallines{0}
\begin{center}
\begin{minipage}[c]{0.8\textwidth}
\gregorioscore{\subfix{nocturnale-romanum/gabc/rhythmica_fr}}
\end{minipage}
\end{center}
}

~

Les rapports entre la \emph{valeur courte} et la \emph{valeur syllabique}, 
et entre la \emph{valeur syllabique} et la \emph{valeur longue}, doivent être cohérents entre eux, mais sont à la main du chantre en fonction de l'acoustique du lieu.

La nature des notes devrait être aussi prise en compte: 
une \emph{stropha} épisémée prend une valeur syllabique plus légère que celle d'une \emph{virga} épisémée, 
la \emph{stropha} étant par nature plus légère que la \emph{virga}.

Enfin, le chant s'élargit naturellement, tout en diminuant de volume, à la fin d'une phrase textuelle ou musicale. C'est ce qui est indiqué, selon la coutume, par le point \emph{mora}.

}

\switchcolumn*
\begin{english}

\intermediatetitle{On critical restitution}

{\setlength{\parindent}{5mm}\small

This book is decidedly a practical edition. It attempts to be critically informed in the following ways.

The neumes indicated above the staff (in versions of this book that have them) are from the Hartker Antiphonary, and in some very rare cases from other Antiphonaries using St.\ Gall notation with rhythmic indications. Antiphons and Invitatories absent from this manuscript have been given synthetic neumes between brackets.

In some rare cases, when the manuscripts agree on a text that is not markedly different from that found in the Roman Breviary, this text was used instead of that found in the  Breviary, as was done for the day hours in the 1912 Roman Antiphonary.

}

\end{english}

\switchcolumn

\intermediatetitle{Restitution et démarche critique}

{\setlength{\parindent}{5mm}\small

Cet ouvrage est fondamentalement une édition pratique. Il adopte une démarche critique de deux manières:

Les neumes imprimés au-dessus de la portée (s'ils sont présents) sont ceux de l'antiphonaire de Hartker, et dans quelques cas très rares, ceux d'autres antiphonaires sangalliens avec notation rythmique. Les antiennes et invitatoires absents de ce manuscrit ont reçu des neumes synthétiques entre crochets.

Dans quelques cas, quand les manuscrits donnent tous un texte qui n'est que légèrement différent du texte du bréviaire romain, ce texte a été conservé au lieu de celui du bréviaire, comme ce fut le cas pour les heures diurnes dans l'antiphonaire romain de 1912.

}

\switchcolumn*

\vspace{\baselineskip}
\sep

\switchcolumn

\vspace{\baselineskip}
\sep

\switchcolumn*

\vfill

\intermediatetitle{Tabella neumatum}

{\gresetnabc{1}{visible}
\gresetclef{invisible}
\gresetinitiallines{0}
\gregorioscore{\subfix{nocturnale-romanum/gabc/neumata}}
}

\vfill

\pagebreak

\switchcolumn

\vfill

{\footnotesize

\intermediatetitle{Editores}

\begin{center}
Matthias Bry

don Jakob Moussong
\end{center}

\intermediatetitle{Socii}

\begin{center}
John Anderson

Dominique Crochu

Gerhard Eger

Dom Giacomo Frigo

Dominique Gatté

Dom Jacques-Marie Guilmard

Alan Liang
\end{center}

}

\vfill

\end{paracol}

% TODO : credit Alan Liang, 


\end{document}